\chapter{Introduction}

The embedded industry is currently undergoing a fundamental change. Nearly all of the big semiconductor corporations are moving away from the traditional 8-bit cores (8051, AVR, MSP430 etc.), and over to more advanced 32-bit ARM based cores. The marked is in demand of connectivity and processing power, areas where the 8-bit cores struggles. One downside of this migration, is that the complexity of designing embedded software increases. The 8-bit cores were so simple to program that an operating system was unnecessary in most cases, while high level application programming interfaces (API)'s are almost essential for a 32-bit architecture. As a result of this, we are beginning to see more and more embedded systems running operating systems. This is great news for the designer, since it makes it much easier to develop complex applications. However, the introduction of high level API's has it's downsides. Designers have less control over whats going on under all the software layers, giving them less flexibility to tailor their applications to suit their specific needs, which in essence is what embedded systems are all about.

The report explores the concept of running a Linux distribution on a relatively small embedded system. The report elaborates on the pros and cons with using such a system to design a simple game of Pong. As well as some aspects in regards to power consumption. The results show the developer can significantly reduce the overall development time 


